\documentclass[a4paper]{report}
\usepackage[margin=2cm]{geometry}
\usepackage[italian]{babel}
\usepackage{amsmath}
\usepackage{amssymb}
\usepackage{amsthm}
\usepackage[ruled, boxed]{algorithm2e}
\usepackage{algpseudocode}
\usepackage[utf8]{inputenc}
\usepackage{graphicx}
\usepackage{float}
\usepackage{booktabs}
\usepackage[table,xcdraw]{xcolor}

\begin{document}

\chapter{Protocollo messaggi}
Per ogni azione del client è definito un protocollo per lo scambio dei messaggi, sia per comunicazioni \textbf{N2N} (\textit{node-to-node}) che \textbf{N2S} (\textit{node-to-server}).

\section{Node To Server}

Questa tipologia di comunicazione vede come entità coinvolte il client del giocatore e il server REST deputato alla gestione delle partite.\newline
Il server REST espone le proprie risorse all'URL \textbf{http://server\_root:8080/server\_war\_exploded/match/}
\begin{table}[h]
	\centering
	\caption{Risorse REST}
	\label{rest-endpoint}
	{\renewcommand{\arraystretch}{1.8}
	\begin{tabular}{@{}lllll@{}}
		\toprule
		\rowcolor[HTML]{FFCCC9}
		Risorsa        & Percorso          & Metodo REST & Input                                                                      & Output                                                                              \\ \midrule
		Dettagli match & /(matchId)/info   & GET         & String matchId                                                             & \begin{tabular}[c]{@{}l@{}}ok(Match)\\ noContent()\end{tabular}                     \\ \midrule
		Lista match    & /                 & GET         &                                                                            & ok(List\textless Match\textgreater)                                                  \\ \midrule
		Crea match     & /                 & POST        & Match match                                                                & \begin{tabular}[c]{@{}l@{}}notAcceptable()\\ created()\\ notModified()\end{tabular} \\ \midrule
		Entra in match & /(matchId)/join   & POST        & \begin{tabular}[c]{@{}l@{}}String matchId\\ String playerJson\end{tabular} & \begin{tabular}[c]{@{}l@{}}ok(Match)\\ badRequest()\end{tabular}                    \\ \midrule
		Lascia match   & /(matchId)/leave  & POST        & \begin{tabular}[c]{@{}l@{}}String matchId\\ String playerJson\end{tabular} & \begin{tabular}[c]{@{}l@{}}noContent()\\ ok()\end{tabular}                          \\ \midrule
		Rimuovi match  & /(matchId)/delete & DELETE      & String matchId                                                             & \begin{tabular}[c]{@{}l@{}}ok()\\ notFound()\end{tabular}                           \\ \bottomrule
	\end{tabular}	
	}
\end{table}
\clearpage

\section{Node To Node}
Le comunicazioni N2N avvengono all'interno di una rete \textbf{Token Ring} sfruttando il protocollo di trasporto \textbf{TCP}.

\subsection{Modello messaggi}
Per modellare i messaggi scambiati tra i vari nodi della rete è stata definita la classe \textbf{RingMessage}. Ogni messaggio è caratterizzato dalle seguenti proprietà:
\begin{itemize}
	\item \textit{MessageType} \textbf{type};
	\item \textit{String} \textbf{sourceAddress};
	\item \textit{String} \textbf{id};
	\item \textit{String} \textbf{content};
\end{itemize}
\textbf{\underline{Nota}:} il campo sourceAddress segue la convenzione \textit{ip:port}.
La tipologia di messaggio viene definita nell'enumerazione \textbf{MessageType} che comprende le seguenti tipologie:
\begin{itemize}
	\item \textbf{ACK} - per messaggi di conferma;
	\item \textbf{GAME} - per messaggi riguardanti azioni relative al funzionamento del gioco;
	\item \textbf{TOKEN} - per messaggi inerenti la gestione del token;
\end{itemize}

\subsection{Messaggi ACK}
I messaggi \textbf{ACK}(\textit{Acknowledgement}) vengono utilizzati da un nodo ricevente per confermare la ricezione di un messaggio al relativo mittente: il messaggio ACK ha lo stesso \textbf{id} del messaggio da confermare.\newline
I messaggi ACK hanno tutti il formato definito nella seguente tabella.

\begin{table}[H]
	\centering
	\caption{Formato messaggi ACK}
	\label{format-ack}
	{\renewcommand{\arraystretch}{1.8}
	\begin{tabular}{@{}lllll@{}}
		\toprule
		\rowcolor[HTML]{FFCCC9} 
		{\color[HTML]{333333} Azione} & {\color[HTML]{333333} Type} & {\color[HTML]{333333} Source} & {\color[HTML]{333333} Id} & {\color[HTML]{333333} Content} \\ \midrule
		Conferma ricezione            & ACK                         & $M_{in}$                  & $M_{in}$              & null                           \\ \bottomrule
		\small{$M_{in}$ si riferisce al messaggio ricevuto}
	\end{tabular}
	}
\end{table}

\subsection{Messaggi TOKEN}
I messaggi \textbf{TOKEN} vengono utilizzati dai nodi della rete per gestire il movimento del token all'interno della rete: mediante un messaggio di questo tipo viene data la possibilità ad un nodo di cedere il token.\newline
I messaggi TOKEN hanno tutti il formato definito nella seguente tabella.

\begin{table}[H]
	\centering
	\caption{Formato messaggi TOKEN}
	\label{format-token}
	{\renewcommand{\arraystretch}{1.8}
		\begin{tabular}{@{}lllll@{}}
			\toprule
			\rowcolor[HTML]{FFCCC9} 
			{\color[HTML]{333333} Azione} & {\color[HTML]{333333} Type} & {\color[HTML]{333333} Source} & {\color[HTML]{333333} Id} & {\color[HTML]{333333} Content} \\ \midrule
			Conferma ricezione            & TOKEN                         & $M$                  & $M$              & null                           \\ \bottomrule			
		\end{tabular}
	}
\end{table}
\clearpage
\subsection{Messaggi GAME}
I messaggi \textbf{GAME} vengono utilizzati per la trasmissione di informazioni relative ad eventi del gioco.\newline
I messaggi GAME hanno i formati definiti nella seguente tabella.

\begin{table}[H]
	\centering
	\caption{Formato messaggi GAME}
	\label{format-game}
	{\renewcommand{\arraystretch}{1.8}
		\begin{tabular}{@{}lll@{}}
			\toprule
			\rowcolor[HTML]{FFCCC9} 
			{\color[HTML]{333333} Azione} & {\color[HTML]{333333} Type}	& {\color[HTML]{333333} Content} \\ \midrule
			Entrata in partita            & GAME	& ENTER-PLAYER\#$Player_{JSON}$	\\ \bottomrule	
			Uscita dalla partita            & GAME	& EXIT-PLAYER\#$Player_{JSON}$	\\ \bottomrule	
			Verifica posizione giocatore            & GAME	& CHECK-POSITION	\\ \bottomrule
			Notifica posizione giocatore            & GAME	& CHECK-POSITION\#$Position_{JSON}$	\\ \bottomrule
			Notifica movimento           & GAME	& MOVE\#$Player_{JSON}$\#$Position_{JSON}$	\\ \bottomrule
			Notifica giocatore ucciso            & GAME	& KILLED\#$Player_{JSON}$	\\ \bottomrule
			Rilascio bomba            & GAME	& BOMB-RELEASE\#$Bomb_{JSON}$	\\ \bottomrule
			Esplosione bomba            & GAME	& BOMB-EXPLOSION\#$Bomb_{JSON}$	\\ \bottomrule
			Uccisione bomba            & GAME	& BOMB-KILL\#$Player_{JSON}$	\\ \bottomrule
			Fine partita            & GAME	& GAME-END	\\ \bottomrule
		\end{tabular}
	}
\end{table}
	
\end{document}